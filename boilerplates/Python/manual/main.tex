%%%%%%%%%%%%%%%%%%%%%%%%%%%%%%%%%%%%%%%%%
% Diaz Essay
% LaTeX Template
% Version 2.0 (13/1/19)
%
% This template originates from:
% http://www.LaTeXTemplates.com
%
% Authors:
% Vel (vel@LaTeXTemplates.com)
% Nicolas Diaz (nsdiaz@uc.cl)
%
% License:
% CC BY-NC-SA 3.0 (http://creativecommons.org/licenses/by-nc-sa/3.0/)
%
%%%%%%%%%%%%%%%%%%%%%%%%%%%%%%%%%%%%%%%%%

%----------------------------------------------------------------------------------------
%	PACKAGES AND OTHER DOCUMENT CONFIGURATIONS
%----------------------------------------------------------------------------------------

\documentclass[11pt]{vantage6} % Font size (can be 10pt, 11pt or 12pt)

%----------------------------------------------------------------------------------------
%	TITLE SECTION
%----------------------------------------------------------------------------------------

\title{\textbf{[Algorithm name]} \\ {\Large\itshape [Short description]}} % Title and subtitle

\author{\textbf{[Thor A.U., ... ]} \\ \textit{[Organization]}} % Author and institution

\date{\today} % Date, use \date{} for no date

%----------------------------------------------------------------------------------------

\begin{document}
\thispagestyle{firstpage}
\vantage6logo
\vspace{5cm}
\maketitle % Print the title section
\pagebreak

%----------------------------------------------------------------------------------------
%	INTRODUCTION
%----------------------------------------------------------------------------------------

\tableofcontents
\pagebreak

%----------------------------------------------------------------------------------------
%	INTRODUCTION
%----------------------------------------------------------------------------------------

\section{Introduction}
% Introduction

%----------------------------------------------------------------------------------------
%	MATHMATICS
%----------------------------------------------------------------------------------------

\section{Mathematics}
\subsection{Central}
% Briefly explain the mathematics in a central setting

\begin{equation}
f(x) = \frac{1}{\sum_i{a_i}}
\end{equation}


\subsection{Federated}
[Explain the federated derivation]

%----------------------------------------------------------------------------------------
%	IMPLEMENTATION
%----------------------------------------------------------------------------------------

\section{Implementation}
\subsection{Parameters}

\begin{tabular}{|l||l|l|l|}

 \hline
 \multicolumn{4}{|c|}{Input Parameters} \\
 \hline
 Parameter 	& type  	& example 	& description \\
 \hline
 a			& string    & "a" 		&  a value, with a extra long description\\
 b 			& int 		& 1 		&  b value\\
 c			& float		& 1.1 		&  ...\\
 \hline
\end{tabular}

\subsection{Algorithm}
\begin{algorithm}
\caption{master}\label{alg:cap}
\begin{algorithmic}
\Require $n \geq 0$
\Ensure $y = x^n$
\State $y \gets 1$
\State $X \gets x$
\State $N \gets n$
\While{$N \neq 0$}
\If{$N$ is even}
    \State $X \gets X \times X$
    \State $N \gets \frac{N}{2}$  \Comment{This is a comment}
\ElsIf{$N$ is odd}
    \State $y \gets y \times X$
    \State $N \gets N - 1$
\EndIf
\EndWhile
\end{algorithmic}
\end{algorithm}

\subsection{Output}
[table of algorithm output(s)]

%----------------------------------------------------------------------------------------
%	RISKS
%----------------------------------------------------------------------------------------

\section{Risks}
% Overview of what is shared with whom. And possible security issues

\begin{enumerate}
\item Issue 1
\item issue 2
\end{enumerate}

%----------------------------------------------------------------------------------------
%	VALIDATION
%----------------------------------------------------------------------------------------

\section{Validation}
% Compare central vs federated results, preferably public dataset]
\lstinputlisting[language=Python]{snippets/snip.py}

%----------------------------------------------------------------------------------------
%	EXAMPLES
%----------------------------------------------------------------------------------------

\section{Examples}
[Preferable multiple examples of how to run it from R, python and a plain API call]
\lstinputlisting[language=R]{snippets/R_example.R}
\lstinputlisting[language=Python]{snippets/snip.py}

%----------------------------------------------------------------------------------------
%	BIBLIOGRAPHY
%----------------------------------------------------------------------------------------

\bibliographystyle{unsrt}

\bibliography{references.bib}

%----------------------------------------------------------------------------------------

\end{document}
